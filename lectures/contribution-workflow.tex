\documentclass[%
% aspectratio=1610
% aspectratio=169
% aspectratio=149
% aspectratio=141
% aspectratio=54
% aspectratio=32
]{beamer}

%\usetheme{pymor}                      % use this for section-based progress bar
\usetheme[globalbar, no-section-outline]{pymor}          % use this for global  progress bar
%\usetheme[staticbar]{pymor}          % use this for fixed "progress" bar
%\usetheme[no-section-outline]{pymor} % use to suppress automatic outline slides
                                      % this option can be combined with the
                                      % above

\usepackage{booktabs}
% \usepackage[scale=2]{ccicons}
\usepackage{minted}
\usepackage{pgfplots}
\usepgfplotslibrary{dateplot}

\usepackage{tikz}
\usetikzlibrary{calc,positioning}

\usepackage{algorithm2e}

\usemintedstyle{trac}

\hypersetup{
	colorlinks=true,
	linkcolor=.
}

\title[pyMOR Contribution Workflow]{pyMOR Contribution Workflow}
\subtitle{}

\date{October 8, 2021 -- Morning session}
\author[Linus Balicki, Hendrik  Kleikamp]{Linus Balicki, Hendrik Kleikamp}
\institute{Institute for Analysis and Numerics - WWU Münster}

\begin{document}

\maketitle
{%
  \setbeamercolor{background canvas}{bg=pymor_blue}
  \setbeamertemplate{frametitle}[plain]
  \setbeamercolor{subsection in toc}{fg=PaleGray, bg=pymor_blue}
  \setbeamertemplate{subsection in toc}
  {\leavevmode\leftskip=2em$\bullet$\hskip1em\inserttocsubsection\par}
 
%\begin{frame}[plain]{Outline}
%  \LARGE\tableofcontents
%\end{frame}
}

\definecolor{green}{HTML}{31a031}
\definecolor{red}{HTML}{a03131}

\def\vdist{1.45cm}
\def\hdist{2.8cm}

\def\customgray{lightgray}

%%%%%%%%%%%%%%%%%%%%%%%%%%%%%%%%%%%%%%%%%%%%%%%%%%%%%%%%%%%%%%%%%%%%%%%%%%%%%%%%
%%%%%%%%%%%%%%%%%%%%%%%%%%%%%%%%%%%%%%%%%%%%%%%%%%%%%%%%%%%%%%%%%%%%%%%%%%%%%%%%
\section{pyMOR GitHub repository}
%%%%%%%%%%%%%%%%%%%%%%%%%%%%%%%%%%%%%%%%%%%%%%%%%%%%%%%%%%%%%%%%%%%%%%%%%%%%%%%%
\begin{frame}[fragile]
	\frametitle{Communicating about pyMOR on GitHub}

	\begin{itemize}
		\item \structure{Discussions}
			\begin{itemize}
				\item How to?
				\item Should I open a PR to add this?
				\item Why is my code not working as expected?
				\item Feature requests
			\end{itemize}
		\item \structure{Issues}
			\begin{itemize}
				\item Bug reports
				\item Things that are missing
				\item Reminders for things that should be done
			\end{itemize}
		\item \structure{Pull requests}
			\begin{itemize}
				\item New features
				\item Bug fixes
				\item Changes to existing functionality
			\end{itemize}
	\end{itemize}
\end{frame}
%%%%%%%%%%%%%%%%%%%%%%%%%%%%%%%%%%%%%%%%%%%%%%%%%%%%%%%%%%%%%%%%%%%%%%%%%%%%%%%%
\section{Contribution workflow}
%%%%%%%%%%%%%%%%%%%%%%%%%%%%%%%%%%%%%%%%%%%%%%%%%%%%%%%%%%%%%%%%%%%%%%%%%%%%%%%%
\begin{frame}[fragile]
	\frametitle{Contributing to pyMOR in 10 steps}

	\begin{enumerate}
		\item Fork the pyMOR repository.
		\item Checkout a new branch in your fork.
		\item Create a virtual environment (\mintinline{shell}{python3 -m virtualenv venv}).
		\item Install pyMOR in editable mode (\mintinline{shell}{pip install -e .[full]}).
		\item ***Add your code here (now)***
		\item Add tests for your code (maybe also a demo or even a tutorial) and run the tests locally (\mintinline{shell}{make docker_test} or \mintinline{shell}{make test}).
		\item Open a pull request on GitHub.
		\item Add a label (pr:change, pr:deprecation, pr:fix, pr:new-feature, pr:removal).
		\item Ask for reviews.
		\item Make sure that the CI pipeline passes (see next slide).
	\end{enumerate}
\end{frame}
%%%%%%%%%%%%%%%%%%%%%%%%%%%%%%%%%%%%%%%%%%%%%%%%%%%%%%%%%%%%%%%%%%%%%%%%%%%%%%%%
\section{Continuous Testing / Integration Setup}
%%%%%%%%%%%%%%%%%%%%%%%%%%%%%%%%%%%%%%%%%%%%%%%%%%%%%%%%%%%%%%%%%%%%%%%%%%%%%%%%
\begin{frame}[fragile]
	\frametitle{Some of the components of pyMOR's CI pipeline}
	
	\begin{itemize}
		\item Linter checks.
		\item Checks for autosquashing of commits, PR labels, and merge conflicts.
		\item Tests in \mintinline{shell}{src/pymortests/} and demos in \mintinline{shell}{src/pymordemos/} are executed in several different settings.
		\item Tutorials in \mintinline{shell}{docs/source/} are used as additional tests.
		\item Documentation (available for inspection under \href{https://docs.pymor.org/list.html}{https://docs.pymor.org/list.html}) and wheels are built.
		\item Installation from different sources (wheels, from source, docker) is performed.
	\end{itemize}
\end{frame}
%%%%%%%%%%%%%%%%%%%%%%%%%%%%%%%%%%%%%%%%%%%%%%%%%%%%%%%%%%%%%%%%%%%%%%%%%%%%%%%%
\section{Important!}
%%%%%%%%%%%%%%%%%%%%%%%%%%%%%%%%%%%%%%%%%%%%%%%%%%%%%%%%%%%%%%%%%%%%%%%%%%%%%%%%
\begin{frame}[fragile]
	\frametitle{Important things to remember when contributing}

	\begin{itemize}
		\item Format your code according to \href{https://www.python.org/dev/peps/pep-0008/}{PEP8} (\href{https://flake8.pycqa.org/en/latest/}{flake8} and \href{https://pre-commit.com/}{pre-commit} hooks can help you).
		\item Document your code properly (to keep the good quality of our \href{https://docs.pymor.org/latest/autoapi/index.html}{API Reference}).
		\item Write useful commit messages, give keywords if possible, for instance \mintinline{text}{[parameters] fix Mu.to_numpy for empty Mu}
		\item Provide a meaningful description for your pull request.
		\item Make sure that all tests pass (otherwise your pull request cannot be merged).
	\end{itemize}
\end{frame}
%%%%%%%%%%%%%%%%%%%%%%%%%%%%%%%%%%%%%%%%%%%%%%%%%%%%%%%%%%%%%%%%%%%%%%%%%%%%%%%%
\section{Contributor acknowledgement}
%%%%%%%%%%%%%%%%%%%%%%%%%%%%%%%%%%%%%%%%%%%%%%%%%%%%%%%%%%%%%%%%%%%%%%%%%%%%%%%%
\begin{frame}[fragile]
	\frametitle{Acknowledgement for your contributions}

	\begin{itemize}
		\item Every contributor will be mentioned on GitHub and in release notes.
		\item Copyright holders are ``pyMOR developers and contributors''.
		\item Everybody can become a main developer.
		\begin{itemize}
			\item Guide the future development of pyMOR.
			\item Direct push access to the GitHub repository.
		\end{itemize}
	\end{itemize}
\end{frame}
%%%%%%%%%%%%%%%%%%%%%%%%%%%%%%%%%%%%%%%%%%%%%%%%%%%%%%%%%%%%%%%%%%%%%%%%%%%%%%%%
\section*{Useful links}
%%%%%%%%%%%%%%%%%%%%%%%%%%%%%%%%%%%%%%%%%%%%%%%%%%%%%%%%%%%%%%%%%%%%%%%%%%%%%%%%
\begin{frame}[fragile]
	\frametitle{Useful links}
	
	\begin{itemize}
		\item \structure{pyMOR homepage:} \href{https://pymor.org/}{https://pymor.org/}
		\item \structure{pyMOR documentation:} \href{https://docs.pymor.org/latest/index.html}{https://docs.pymor.org/latest/index.html}
		\item \structure{Developer documentation:} \href{https://docs.pymor.org/latest/developer_docs.html}{https://docs.pymor.org/latest/developer\_docs.html}
		\item \structure{GitHub repository:} \href{https://github.com/pymor/pymor}{https://github.com/pymor/pymor}
		\item \structure{Discussions:} \href{https://github.com/pymor/pymor/discussions}{https://github.com/pymor/pymor/discussions}
		\item \structure{Issues:} \href{https://github.com/pymor/pymor/issues}{https://github.com/pymor/pymor/issues}
		\item \structure{Pull requests:} \href{https://github.com/pymor/pymor/pulls}{https://github.com/pymor/pymor/pulls}
	\end{itemize}
\end{frame}
\plain{Are there questions?}
\end{document}

%%% Local Variables:
%%% mode: latex
%%% TeX-master: t
%%% End:
